%% LyX 2.1.2 created this file.  For more info, see http://www.lyx.org/.
%% Do not edit unless you really know what you are doing.
\documentclass[11pt]{article}
\usepackage[utf8]{luainputenc}
\usepackage[a4paper]{geometry}
\geometry{verbose}
\usepackage{amsmath}

\makeatletter

%%%%%%%%%%%%%%%%%%%%%%%%%%%%%% LyX specific LaTeX commands.
%% Because html converters don't know tabularnewline
\providecommand{\tabularnewline}{\\}

%%%%%%%%%%%%%%%%%%%%%%%%%%%%%% User specified LaTeX commands.
\usepackage{graphics}
\usepackage{graphicx}
\usepackage{subcaption}
\usepackage{caption}
\usepackage{listings}
\usepackage{multirow}
\usepackage{url}
\usepackage[labelfont=bf]{caption}

\makeatother

\usepackage{listings}
\renewcommand{\lstlistingname}{Listing}

\begin{document}
%%Title

\begin{center}
{\Large{}\hrulefill{}}\\
{\Large{} }\textbf{\Large{}Software}{\Large{} }\\
{\Large{} GF2 - First Interim Report }\\
{\Large{} \vspace{0.2cm}
 }SOFTWARE DESIGN TEAM 1 \\
 Quang-Thinh Ha - \textit{qth20}; Edgar Dakin - \textit{ed408}; Konstantinos
Kyriakopoulos - \textit{kk492} \\
 \hrulefill{}\\
 
\par\end{center}




\section{Introduction and General Approach}

The aim of this project is to develop a logic simulation program in
C++. The project involves all five major phases of the software engineering
life cycle: specification, design, implementation, testing and maintenance. \\
\noindent
This report will include a general approach from the team, including
teamwork planning and our agreed EBNF for syntax, along with analysing
all possible semantic errors and errors handling. \\
\noindent
The system is named Elea, after the Ancient Greek colony of Ἐλέα, home to early logicians such as Parmenides and Zeno.


\section{Teamwork Planning}

\noindent The following tasks have been allocated to the members of
the team, with their corresponding deadlines for the remaining allowance
time of the project.

\begin{center}
\begin{tabular}{lll}
\textbf{Week}  & \textbf{Member}  & \textbf{Activity} \tabularnewline
\textit{2}  & Edgar (E)  & Semantic analysis and error handling. \tabularnewline
 & Konstantinos (K)  & Finalise EBNF for syntax. \tabularnewline
  & Quang (Q)  & Start GUI design. \tabularnewline
\textit{3}  & E and K  & Design and implement \texttt{names, scanner, parser classes}.\tabularnewline
  & Q  & Design and implement \texttt{gui classes}\tabularnewline
\textit{4}  & E, K and Q  & Integration and final testing. \tabularnewline
  &   & Maintenance. \tabularnewline
\end{tabular}
\par\end{center}


\section{Language Description}


\subsection{Syntax}

The circuit description language, named Nyaya\footnote{from the Sanskrit word न्याय meaning logic},
is to consist of 5 high level commands, each beginning with an identifying
keyword and terminating in a semi-colon, for robust LL(1) parsing.
\begin{description}
\item [{CREATE:}] Defines and names the individual devices in the circuit,
specifying relevant parameters for each one. A single CREATE statement
can create multiple devices, delimiting devices of the same type by
commas and different types by ampersands. The declaration of each
device is reminiscent of constructors in object-oriented programming
languages, taking the form $type\, name\,(comma-delimited\, parameters)$.
\item [{CONNECT:}] Creates a connection between an output pin and an input
pin.
\item [{DEFINE:}] Defines a new type of device containing instances of
previously defined types, added using CREATE and connected using CONNECT,
and accepting specified parameters, which can be passed to the internal
devices. Pins of the internal devices are selected as inputs and outputs
of the new device.
\item [{MONITOR:}] Specifies output pins to be monitored.
\item [{INCLUDE:}] Links to a library file containing additional DEFINE
commands to be parsed before parsing the rest of the current file.
\end{description}
The complete syntax of Nyaya in EBNF follows below.

\begin{lstlisting}
nyayagrammar = { include | create | connect | define | monitor };
libgrammar = { define };
include = "INCLUDE", lib, ";";
create = "CREATE", typecreate, {"&", typecreate}, ";";
typecreate = type, init, {",", init};  
init = name, "(", [ param, {",", param} ], ")";
param = int_param | ref_param;
connect = "CONNECT", in, "->", out, ";";
in = device, [".", input]; out = device, [".", output];
define = "DEFINE", new_type, "(", [param_name, {",", param_name, }, ")", "{", create, { create | connect }, inputs, outputs }, "}";
inputs = "INPUT", increate, {"," , increate}; outputs = "OUTPUT", outcreate, {"," , outcreate};
increate = in_name, "=", in; outcreate = out_name, "=", out;
monitor = "MONITOR", out, {"," , out};
\end{lstlisting}



\subsection{Semantics}

The following semantic constraints apply to the language and must
be validated when a Nyaya file is read in. 
\begin{itemize}
\item \textit{lib} must point to a file which is a valid libgrammar
\item \textit{type} must correspond to the name of a primitive type or a
type previously defined
\item \textit{name} must be an alphanumeric string that is not already the
name of a device in the current context (ie. within a given define
or outside defines)
\item the number of \textit{param} in init must equal that required by the
type
\item \textit{int\_param }must be an integer which satisfies the validation
required by the type \textit{ref\_param }must correspond to one of
the \textit{param\_names }within the current define statement
\item \textit{device} must be the name of one of the devices previously
created within the current context (ie. within a given DEFINE or outside
DEFINEs) input must be present unless the device has exactly one input
input must be one of the input names for that device (eg. I1, I3,
data, clock) output must be present unless the device has exactly
one output output must be one of the output names for that device
(e.g. Q, Qbar)
\item \textit{new\_type} must be an alphanumeric string that is not already
the name of a primitive type or a type previously defined \textit{param\_name}
must be an alphanumeric string that does not correspond to a previous
\textit{param\_name }in the same define
\item \textit{in\_name }must be an alphanumeric string that does not correspond
to a previous \textit{in\_name}
\item \textit{out\_name} must be an alphanumeric string that does not correspond
to a previous \textit{out\_name}
\end{itemize}

\section{Semantic Errors and Handling}
\end{document}
